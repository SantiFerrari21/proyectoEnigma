\section{Historias de usuarios (\textit{Product backlog})}
\label{sec:backlog}


%Descripción: En esta sección se deben incluir las historias de usuarios y su ponderación (\textit{history points}). Recordar que las historias de usuarios son descripciones cortas y simples de una característica contada desde la perspectiva de la persona que desea la nueva capacidad, generalmente un usuario o cliente del sistema. La ponderación es un número entero que representa el tamaño de la historia comparada con otras historias de similar tipo.
Se muestran a continuación, las historias de usuarios recopiladas. Al principio se presentan dos épicas y sus correspondientes desgloses emanadas de las máximas autoridades de la empresa.

La ponderación de estas historias se realizó teniendo en cuenta la complejidad y el tiempo necesario en resolverlas (criterios complejidad-volumen). Se eligió para ello una escala basada en la serie de Fibonacci desde el 0 al 13, donde el 0 representa poco esfuerzo y el 13 alto esfuerzo para lograr el objetivo. 

La siguiente tabla muestra la clasificación de las funcionalidades observadas y su ponderación.


\begin{table}[ht]
%caption{Tabla de ponderación}
\begin{tabularx}{\linewidth}{@{}|l|X|X|l|@{}}
\hline
\rowcolor[HTML]{C0C0C0} 
Funcionalidad solicitada           & Ponderación 	\\ \hline

Visualización de temperatura & 0\\ \hline
Gráficas en tiempo real & 1\\ \hline
Gráficas históricas, alarmas, estados & 3\\ \hline
Mapas, seguridad & 5\\ \hline
Fuera del alcance de este proyecto & 13\\ \hline

\end{tabularx}
\end{table}
Luego, se les asignó una prioridad de resolución, teniendo como prioridad fundamental que el sistema funcione de forma ininterrumpida. Esto se aprecia en la siguiente tabla.

\begin{table}[ht]
%caption{Tabla de prioridades}
%\label{tab:interesados} 
\begin{tabularx}{\linewidth}{@{}|l|X|X|l|@{}}
\hline
\rowcolor[HTML]{C0C0C0}
Funciones           & Prioridad 	\\ \hline
Inherentes a la robustez del sistema & 1 \\ \hline
Inherentes a las notificaciones, registros y configuraciones& 2 \\ \hline
Inherentes a la visualización de datos &3 \\ \hline
Inherentes a la seguridad & 4 \\ \hline
Sin prioridad & 5 \\ \hline
\end{tabularx}
\end{table}

Descripción de las historias de usuarios


\begin{itemize}
\item ÉPICA: como Secretario de Salud Pública Municipal, necesito estar seguro de la buena conservación de las vacunas, para poder informar al estado provincial y nacional la disponibilidad de las mismas.

Desglose
	\begin{itemize}
	\item 
    Como secretario de Salud Pública Municipal deseo poder visualizar en cualquier momento la temperatura de cualquier heladera con el objetivo de asegurar que no se deteriore su contenido. 
    
Ponderación:0 Prioridad:3
	\end{itemize}
	
	\begin{itemize}
	\item 
	Como secretario de Salud Pública Municipal deseo poder recibir una alarma si alguna de las heladeras superan la temperatura de 4 grados centígrados con el objetivo de poder actuar rápidamente y que no se deterioren las vacunas. 
	
Ponderación:3 Prioridad:2
	\end{itemize}

	\begin{itemize}
	\item 
	Como secretario de Salud Pública Municipal deseo poder visualizar en cualquier momento cuántas vacunas hay por heladera con el fin de saber si se pueden reubicar vacunas ante una eventual falla de alguna heladera. 

Ponderación:13 Prioridad:5
	\end{itemize}
\end{itemize}

\begin{itemize}
\item ÉPICA: como Director de Infraestructura Hospitalaria, deseo conocer si algún efector de la red presenta problemas en los refrigeradores críticos y así direccionar las acciones necesarias para su corrección. 

Desglose

	\begin{itemize}
	\item Como Director de Infraestructura Hospitalaria, necesito visualizar mediante un mapa interactivo el estado de todos los refrigeradores para poder diagramar las acciones de corrección necesarias.  

Ponderación:5 Prioridad:3
	\end{itemize}

	\begin{itemize}
	\item Como Director de Infraestructura Hospitalaria, necesito recibir notificaciones de las anomalías de los equipos de refrigeración para poder indagar a los responsables de la situación.  

Ponderación:3 Prioridad:3
	\end{itemize}

\end{itemize}


\begin{itemize}
\item ÉPICA: como Director de Infraestructura Hospitalaria, deseo que a este sistema se le puedan adicionar otras variables físicas para tener un panel de control completo de todo el equipamiento a mi cargo. 

Desglose

	\begin{itemize}
	\item Como Director de Infraestructura Hospitalaria, deseo conocer el nivel de los tanques de agua de todos los efectores de la Municipalidad, para prevenir posibles faltantes de este elemento primordial.
	
Ponderación: 13 Prioridad:5
	\end{itemize}

	\begin{itemize}
	\item Como Director de Infraestructura Hospitalaria, deseo conocer el estado de los filtros HEPA de los equipos de aire acondicionado de todos los efectores de la Municipalidad, para poder programar su compra y recambio.
	
Ponderación: 13 Prioridad:5
	\end{itemize}

\end{itemize}

\begin{itemize}
\item Como responsable de mantenimiento necesito conocer la evolución de las temperaturas de los refrigeradores en los últimos 6 meses, para estimar las intervenciones preventivas en tales equipos. 

Ponderación: 3 Prioridad:3
\end{itemize}


\begin{itemize}
\item Como operario de mantenimiento necesito que el sistema me envíe inmediatamente un alerta a mi teléfono si hay algún apartamiento de temperaturas, para solucionar rápidamente el desperfecto. 

Ponderación:3 Prioridad:2
\end{itemize}


\begin{itemize}
\item Como encargado de mantenimiento electrónico, necesito saber cuándo un sensor deja de funcionar para inmediatamente proceder a su reparación. 

Ponderación:3 Prioridad:1
\end{itemize}

\begin{itemize}
\item Como encargado de mantenimiento electrónico, necesito poder visualizar el último dato de temperatura enviado al servidor para comprobar la completitud de los datos almacenados

Ponderación:3 Prioridad:1
\end{itemize}


\begin{itemize}
\item Como jefe de seguridad informática, necesito que la comunicación de los sensores se haga con mecanismos de encriptación, para proteger los datos de posibles cambios por intromisiones no deseadas. 

Ponderación:5 Prioridad:4
\end{itemize}


\begin{itemize}
\item Como Director de farmacia desearía que el acceso al sistema de visualización sea con usuario y contraseña para poder definir roles de usuarios. 

Ponderación:5 Prioridad:4
\end{itemize}


\begin{itemize}
\item Como director de Bioingeniería, debería tener acceso a los datos históricos de temperatura para evaluar el funcionamiento a largo plazo de los equipos de refrigeración. 

Ponderación:3 Prioridad:3
\end{itemize}

\begin{itemize}
\item Como director de Bioingeniería, desearía poder visualizar una tabla con los históricos de alarmas para evaluar la cantidad de fallos de los equipos de refrigeración. 

Ponderación:3 Prioridad:3
\end{itemize}


\begin{itemize}
\item Como jefa de droguería, necesito tener en pantalla el registro histórico de temperaturas de la última semana para visualizar si los productos perdieron la cadena de frío. 

Ponderación:1 Prioridad:2
\end{itemize}

\begin{itemize}
\item Como jefa de droguería, necesito tener acceso a cambiar los rangos de temperatura para poder almacenar diferentes productos en diferentes refrigeradores según las necesidades. 

Ponderación:3 Prioridad:2
\end{itemize}

\begin{itemize}
\item Como técnico hemoterapista de guardia, necesito obtener el registro de las últimas 24 hs de temperaturas para entregar mi guardia con los productos asegurados en su cadena de frío. 

Ponderación:1 Prioridad:2
\end{itemize}

\begin{itemize}
\item Como operario de droguería, necesito que el sistema registre las temperaturas cada 10 minutos para evitar los olvidos y las imprecisiones del registro manual. 

Ponderación:0 Prioridad:2
\end{itemize}