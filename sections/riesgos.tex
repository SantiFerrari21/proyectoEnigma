\section{10. Gestión de riesgos}
\label{sec:riesgos}
Se describen los riesgos para el desarrollo del proyecto y su plan de mitigación.

\textbf{a) Identificación de los riesgos y estimación de sus consecuencias:}
 
Riesgo 1: El recipiente seleccionado para la conservadora no este suficientemente aislado, llevando esto a que el equipo no sea capaz de mantener su temperatura interna.
\begin{itemize}
\item Severidad (S): 10. El riesgo es máximo, puesto que detendría el funcionamiento del equipo.
\item Ocurrencia (O): 3. Se asigna esta ocurrencia ya que se realizaran las pruebas necesarias para evitar este problema.
\end{itemize}   

Riesgo 2: Falla del firmware.
\begin{itemize}
\item Severidad (S): 7. Este error provocaría un funcionamiento inestable.
\item Ocurrencia (O): 8. Se asigna esta ocurrencia debido a la dificultad del desarrollo de firmware.
\end{itemize}   


\textbf{b) Tabla de gestión de riesgos:}

\begin{table}[htpb]
\centering
\begin{tabularx}{\linewidth}{@{}|X|c|c|c|c|c|c|@{}}
\hline
\rowcolor[HTML]{C0C0C0} 
Riesgo & S & O & RPN & S* & O* & RPN*  \\ \hline
1. La aislación no es la adecuada & 10 & 3 & \cellcolor[HTML]{00cc00} 30 & - &  -  & - \\ \hline
2. Falla de firmware & 7 & 8 & \cellcolor[HTML]{ff0000}56 & 7 &  3  & \cellcolor[HTML]{00cc00}21 \\ \hline
\end{tabularx}%
\end{table}

Criterio adoptado:Se trabajara para mitigar las medidas con un RPN mayor a 40.

Nota: los valores marcados con (*) en la tabla corresponden luego de haber aplicado la mitigación.

\textbf{c) Plan de mitigación de los riesgos que originalmente excedían el RPN máximo establecido:}

Riesgo 2: Se trabajara meticulosamente durante el desarrollo de cada una de las funciones del firmware. Se pondra en prueba el sistema durante un período de prueba hasta lograr estabilidad.
\begin{itemize}
\item Severidad (S): 7. No se modifica.
\item Probabilidad de ocurrencia (O): 3. Con el proceso de pruebas se espera una baja probabilidad de fallas.
\end{itemize}