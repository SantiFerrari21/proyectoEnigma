\documentclass[12pt]{proyecto}
\usepackage{pdflscape}
\usepackage{import}
% El títulos de la memoria, se usa en la carátula y se puede usar el cualquier lugar del documento con el comando \ttitle
\titulo{Conservadora de temperatura para muestras bacteriologicas} 

% Nombre del posgrado, se usa en la carátula y se puede usar el cualquier lugar del documento con el comando \degreename
%\posgrado{Carrera de Especialización en Sistemas Embebidos} 
\posgrado{Carrera de Técnico en Electrónica} 
%\posgrado{Carrera de Especialización en Intelegencia Artificial}
%\posgrado{Maestría en Sistemas Embebidos} 
%\posgrado{Maestría en Internet de las cosas}

% Tu nombre, se puede usar el cualquier lugar del documento con el comando \authorname
\autor{A. Aquino, M. Cerisoli, S. Ferrari, S. Ferreyra} 

% El nombre del director y co-director, se puede usar el cualquier lugar del documento con el comando \supname y \cosupname y \pertesupname y \pertecosupname
\director{Rúben Muñoz}
\pertenenciaDirector{Sexto Electrónica}

% Nombre del cliente, quien va a aprobar los resultados del proyecto, se puede usar con el comando \clientename y \empclientename
\cliente{Fulano Goni}
\empresaCliente{Sexto Industrias y Procesos}

% Nombre y pertenencia de los jurados, se pueden usar el cualquier lugar del documento con el comando \jurunoname, \jurdosname y \jurtresname y \perteunoname, \pertedosname y \pertetresname.
\juradoUno{Marcelo Castello}
\pertenenciaJurUno{Sexto Electrónica} 
%\juradoDos{Nombre y Apellido (2)}
%\pertenenciaJurDos{pertenencia (2)}
%\juradoTres{Nombre y Apellido (3)}
%\pertenenciaJurTres{pertenencia (3)}
 
\fechaINICIO{1 de Junio de 2022}		%Fecha de inicio de la cursada de GdP \fechaInicioName
\fechaFINALPlanificacion{27 de Junio de 2022} 	%Fecha de final de cursada de GdP
\fechaFINALTrabajo{27 de Junio de 2022}		%Fecha de defensa pública del trabajo final


\begin{document}

\maketitle
\thispagestyle{empty}
\pagebreak


\thispagestyle{empty}
{\setlength{\parskip}{0pt}
\tableofcontents{}
}
\pagebreak

%TODO: Enumerar bien las secciones.

\import{sections/}{registro.tex}

\import{sections/}{acta.tex}

\import{sections/}{descripcion.tex} %TODO: Descripción y diagrama de bloques

\import{sections/}{interesados.tex} %TODO: Definir Roles

\import{sections/}{propositos.tex}

\import{sections/}{alcance.tex} %TODO: Definir alcance real del proyecto

\import{sections/}{supuestos.tex}

%\import{sections/}{requerimientos.tex} %TODO: No se que hay que poner acá

%\import{sections/}{backlog.tex}

\import{sections/}{entregables.tex} %TODO: Considerar más entregables

\import{sections/}{wbs.tex} %TODO: Que hacer?...

\import{sections/}{recursos.tex}

\import{sections/}{presupuesto.tex}
 
%\import{sections/}{responsabilidades.tex}

\import{sections/}{riesgos.tex} %TODO: Pensar más riesgos

\import{sections/}{calidad.tex} %TODO: Definir y escribir la gestión de calidad

%\import{sections/}{comunicaciones.tex}

%\import{sections/}{compras.tex} %TODO: Definir proceso de compras | Este punto quiza no sea necesario

%\import{sections/}{seguimiento.tex}

%\import{sections/}{cierre.tex} %No hay mucho que hacer en este apartado

\end{document}